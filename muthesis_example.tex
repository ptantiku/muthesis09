\documentclass[natbib]{muthesis09}

% IMPORTANT!!! -- DO NOT USE THIS FILE FOR YOUR THESIS --
% make a COPY of muthesis_template.tex instead

%\draft
%\notitlepage
\usepackage{graphicx}
\newcommand{\beq}{\begin{equation}}
\newcommand{\eeq}{\end{equation}}
\newcommand{\bfig}{\begin{figure}}
\newcommand{\efig}{\end{figure}}
\newcommand{\bfa}{\left\{\begin{array}}
\newcommand{\efa}{\end{array}\right.}
%\renewcommand\textfraction{.01}
%\setcounter{totalnumber}{9}
%\setlength{\floatsep}{1cm} 
%\setcounter{secnumdepth}{4} \setcounter{tocdepth}{4}
\renewcommand{\arraystretch}{1.5}

\bibliographystyle{uwthesis} % if you are using bibtex

% information for front page

\title{Dynamics of Spiral Waves under Feedback Control in the
Light-Sensitive Belousov-Zhabotinsky Reaction} 
\candidate{Onuma Kheowan}
\degree{PhD}
\subject{Chemical~Physics} 
\submissionyear{2001}
%\isbn{974-04-0357-3}

% information for page i (advisors)

\candidatetitle{Miss.\ }
\majoradvisor{Tata Young}
\majoradvisortitle{Prof.\ }
\majoradvisorletters{Dr.rer.nat.}
\majoradvisorsubject{Chemistry}
\coadvisor{Alan \mbox{Partridge}}
%\coadvisor{Alan Partridge}
\coadvisortitle{Prof.\ }
\coadvisorletters{Dr.rer.nat.habil.}
\coadvisorsubject{Mechanical Engineering and Operational Management}
\coadvisorstatus{Major advisor}
\coadvisorII{Austin D. Powers}
\coadvisorIItitle{Assoc.\ Prof.\ }
\coadvisorIIletters{Ph.D.}
\coadvisorIIsubject{Theoretical Physics and \mbox{Chemistry}}
\coadvisorIII{Austin D. Flowers}
\coadvisorIIItitle{Prof.\ }
\coadvisorIIIletters{M.Sc.}
\coadvisorIIIsubject{Nanophysics}
\graduatestudiesdean{Prof.\ Liangchai Limlomwongse, Ph.D.}
\GSDstatus{Acting}
\programchair{Assoc.\ Prof.\ Pongtip Winotai}
\programchairqual{Ph.D. (Physical Chemistry)}
\faculty{Information and Communication Technology}

% information for page ii (exam committee)

\submissiondate{4th June 2001}
\chair{Prof. D. Scheitzermann}
\chairqual{Ph.D. (Eugenics)}
\memberIII{Assoc.~Prof.~Pongtip Winotai}
\memberIIIqual{Ph.D. (Chemical Physics)}
\memberIV{Ahpisit Ungkitchanukit}
\memberIVqual{Ph.D. (Chemical Physics and Physical Chemistry)}
\facultydean{Prof.~Amaret Bhumiratana}
\FDqual{Ph.D. (Microbiology)}
\FDstatus{Acting}

% information for page iv (ABSTRACT)

\candidatenumber{3836484 SCCP/D}
%\longsubject
\keywords{spiral wave dynamics / BZ reaction / feedback}
\keywordsII{control / resonance attractor / light-sensitive}
\keywordsIII{\NoCaseChange{Ru}-catalyst / tip trajectory / excitable media}

% information for page v (THAI ABSTRACT)
\thaisubject{ԡԧ}
\thaicandidate{ ҹ}
\thaititle{ʵͧ㹻ԡ٫Ϳ⺷ԧʡշʧ}
%\thaititle{ʵͧäǺẺ͹Ѻ㹻ԡ٫Ϳ⺷ԧʡշʧ}
\thaimajoradvisor{þԹ ѧѹ}
%\thaicoadvisor{}
\thaicoadvisorII{ҷ ѧ}
%\thaicoadvisorIII{ ѧѧ}

% information for Biography

\dateofbirth{16th December 1972}
\placeofbirth{Suphanburi, Thailand}
\firstdegree{Bachelor of Science}
%\longfirstdegree
\firstdegreemajor{Physics}
\firstdegreeinstitution{Kasetsart University}
\firstdegreeyears{1991--1995}
\years{1995--2001}
\postinstitution{Otto-von-Guericke Universit\"{a}t,}
\postinstitutionLnII{Magdeburg, Germany, 1998--2001}
\scholarship{Development and Promotion of Science and }
\scholarshipLnII{Technology Talents Project (DPST),}
\scholarshipLnIII{Thailand, 1991--2000}
\scholarshipII{Deutscher Akademischer Austauschdienst}
\scholarshipIILnII{(DAAD), Germany, 1998--2000}
\scholarshipIII{Postgraduate Education and Research Program}
\scholarshipIIILnII{in Chemistry, 2000--2001}
\position{Lecturer, 1997--Present}
\workplace{Chemistry Department, Mahidol University}
\homeaddress{23/4 Rama 4 Road}
\homeaddressLnII{Bangkok 10330 Thailand}
\email{scokw@mahidol.ac.th}

\begin{document}
\maketitle
\acknowledgements{
I would like to first express my sincere gratitude to my advisor,
Prof.\ Orapin Rangsiman, for her invaluable advice and
encouragement throughout. I am equally grateful to Prof.\ Stefan C.\
M\"{u}ller of Otto-von-Guericke Universit\"{a}t, Magdeburg, 
for his helpful guidance,
supervision, and giving me the opportunity to work in his research group.
My deep gratitude also to Dr.~Vladimir Zykov for his
insightful advice and valuable discussions. I am also thankful to
Dr.\ Ahpisit Ungkitchanukit and Assoc.\ Prof.\ Pongtip Winotai for
their kind comments and support. In addition, I express my special
thanks to Asst.\ Prof.\ Michael A.\ Allen for his comments and
help with \LaTeX.

I gratefully acknowledge the Deutscher Akademischer
Austauschdienst, Sandwich Program (DAAD) for giving me such a great
opportunity to conduct my research at
Otto-von-Guericke Universit\"{a}t, Magdeburg. I am particularly
indebted to the Development and Promotion of Science and
Technology Talents Project (DPST), funded by the Royal Thai
Government for the scholarship which enabled me to undertake this
study. I also wish to thank the 
Postgraduate Education and Research Program in Chemistry,
funded by the Royal Thai Government, for partial support.

I have learnt a lot from the people that I have worked with in the
laboratory of Prof.\ Stefan C.\ M\"{u}ller in Magdeburg during these
years. Words are inadequate to express my thanks to Dirk Michael
Goldschmidt for his expertise and insightful hints for getting
started with the experimental work. Special thanks go to Dr.~Marcus
Hauser and Dr.\ Thomas Mair for their kind advice, support,
comments and critical reading of the manuscript. In addition, I
wish to gratefully acknowledge Dr.~C.\ K.\ Chan, who provided me with
valuable hints and fruitful collaboration. The support from all
other colleagues is also gratefully acknowledged, in particular,
Niklas Manz, Ulrich Storb, Bernd Schmidt, Eric Kasper, Wolfgang
Janto\ss, Katja Guttmann and Ramona Bengsch.

I greatly appreciate all my friends for their kind help on
numerous occasions, in particular, Jareerat Samran, Pakavedee
Sukanan and all others. Finally, I express my sincere gratitude to
my family for their encouragement and support throughout my course
of study.
}

\abstract{
Dynamical properties of rigidly rotating spiral waves under
feedback control are studied experimentally in the light-sensitive
Belousov-Zhabotinsky reaction. After characterizing the influence
of illumination on various spiral parameters, the behaviour of the
spiral waves under both local and global feedback control is
investigated.
In local feedback, short light pulses are applied at the same time as,
or at a fixed time interval after, the passage of a wave front through a
preselected measuring point. It is shown that this type of feedback
results in a drift of the spiral wave core along a discrete set of
stable circular orbits centred at the measuring point.

When the time delay in the feedback loop exceeds the rotation
period of the spiral, the stability of the resonance attractor
gets lost and pronounced deviations of the core trajectories from
circular orbits are observed. In a theory that reduces the spiral
wave dynamics to a higher order iterative map, these deviations
are explained to be a result of instabilities appearing due to the
Neimark bifurcation of the map.

Finally, global aspects were incorporated into the feedback signal
by choosing spatially extended control domains, where the
intensity of the illumination is taken to be proportional to the
average wave activity observed within a circular domain of the
reaction layer. Stabilization and destabilization of spiral waves,
as well as the existence of new types of attractors are
demonstrated.
}

%\thaiabstract{
%Ԩ¹繡֡ѵԷҧʵͧ (spiral
%waves) äǺẺ͹Ѻ (feedback
%control) 㹻ԡ٫Ϳ⺷Թʡ (Belousov-Zhabotinsky
%reaction) ʧ
%¡õǨͺԷԾŢͧʧյͤҺͧع
%Ǥ Ǣͧ˹Ҥ 鹼ҹٹҧͧ᡹ͧ (spiral wave
%core) ѧҡӡ֡´ͧçҧͧṹ͵á
%(resonance attractor) äǺ蹷¡éʧ繪ǧ
%(light pulse) 㹢з͹ҹش¡ҨشѴ (measuring point)
%˹ ʹ¤˹ǧ (time delay) ˹ 
%äǺẺ͹Ѻա͹ (drift) ͧ᡹ͧ (spiral
%wave core) ǧ⤨ (orbit) շҧѹ繪ش
%¹ҵâͧäǺẺ͹Ѻ觼Դâ
%(transition) ҧǧ⤨ ѧ鹨֧äǺ͹ (drift) ͧ᡹ͧ (spiral
%wave core) Ẻ (snail-shaped) ҧشѴ (measuring point)
%᡹ͧ (spiral wave core) ӴѺ
%
%֡ҡäǺ͹ѺẺͺ (global
%feedback) ѭҳͧäǺ鹤ӹdzҨҡ鹷ٻǧͧԡ
%ʴ繶֧ʶҾСʶҾͧ
%駡͵á (attractor) ԴԴ
%š÷ͧѧࡵö͸Ժ¤͸Ժ·ҧԵʵѲҢ
%}

%\linespacing{1.2}
\tableofcontents
%\listoftables
\listoffigures


%\linespacing{1.77}

\chapter{test}

As shown in Ref.~\cite{APR06,MP99,ZK74}, nonlinear waves are interesting.


\chapter[Very long chapter title]
{A very long chapter title that will not fit in the
  space available for the page heading}
\chapter{test}
\chapter{A title with a chemical symbol: \NoCaseChange{NaMnO$_4$}}
\bfig
\vspace{2cm}
\caption{hjk}
\efig
\chapter{test}
\section{test}
\bfig
\vspace{2cm}
\caption{hjk}
\efig
\chapter{test}
\bfig
\vspace{2cm}
\caption{hjk}
\efig
\chapter{test}
\bfig
\vspace{2cm}
\caption{hjk}
\efig
\chapter{test}
\bfig
\vspace{2cm}
\caption{hjk}
\efig
\chapter{test}
\bfig
\vspace{2cm}
\caption{hjk}
\efig
\chapter{test}
\bfig
\vspace{2cm}
\caption{hjk}
\efig
\chapter{test}
\bfig
\vspace{2cm}
\caption{hjk}
\efig
\chapter{test}
\bfig
\vspace{2cm}
\caption{hjk}
\efig
\chapter{test}
\bfig
\vspace{2cm}
\caption{hjk}
\efig
\chapter{test}
\bfig
\vspace{2cm}
\caption{hjk}
\efig
\chapter{test}
\bfig
\vspace{2cm}
\caption{hjk}
\efig
\chapter{test}
\bfig
\vspace{2cm}
\caption{hjk}
\efig
\chapter{final test}
\bfig
\vspace{2cm}
\caption{hjk}
\efig
\newpage\mbox{}
\bfig
\vspace{2cm}
\caption{hjk}
\efig
\newpage\mbox{}
\bfig
\vspace{2cm}
\caption{hjk}
\efig
\newpage\mbox{}
\chapter{test}
\bfig
\vspace{2cm}
\caption{hjk}
\efig
\noindent
fdhasjk
%\input{introduction}
\appendix
%\appendices
\chapter{Meuk}
%\chapter{Meuk2}
Consider the well-known equation
\begin{equation}
E=mc^2.
\end{equation}
\newpage
Another page.
%\chapter{Taeng}
%\input{references_template} % if you are not using bibtex
\bibliography{abbrev,fc,solitons,books} % for bibtex (recommended!!!)
\biography
\end{document}
